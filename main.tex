\documentclass[12 pt, letterpaper]{article}
\usepackage[spanish]{babel}
\usepackage[utf8]{inputenc}
\usepackage[usenames]{color}
\usepackage{graphicx}

\begin{document}

\section{Validacion de resultados}
\subsection{Parte 4.1}
\subsection{Validacion de Resultados}
\subsubsection{Validacion de resultados del indicador porcentaje de Morosidad}
Para validar la hipotesis especifica: "La implementacion de un sistema de informacion basado en un enfoque de procesos, disminuye el porcentaje de morosidad de la MAcrofinanciera CRECER", se realizo una prueba estadistica usando la distribucion t-student para realizar la comparación se tomo datos del primer semestre del año 2012 (con media poblacional igual a 3.436 porciento) y los datos correspondientes al segundo semestre del 2013.
\begin{figure}[H]
    \includegraphics[width=\textwidth]{Imagenes_Parte_4.2/image1.jpeg}
    \caption{Zona de Rechazo y no rechazo de hipotesis (porcentaje de morosidad)}
\end{figure}
En la grafica se aprecia la zona de aceptacion y las zonas de rechazo de la hipotesis nula, dividas por los puntos criticos -2.7765 y 2.7765, tambien se observa el valor del estadístico de prueba(-4.904), ubicado en la zona de rechazo de la Ho del lado izquierdo.
\begin{center}
\centering{-4.904 < -2.7765}
\end{center}
Por ello se toma la decision de 
\textbf{rechazar la hipotesis nula y aceptar la hipotesis alternativa}
, en conclusion se puede decir que existe evidencia suficiente para afirmar que el porcentaje de morosidad disminuyo el segundo semestre del año 2013 con respecto al primer semestre del año 2012.

\subsubsection{Validacion de resultados del indicador tiempo de evaluacion y otorgamiento de creditos}
Al analizar los tiempos de evaluacion de creditos para el año 2012(obtenidos 5 meses antes de la implementacion de COREBANK) se obtuvo un promedio de 46,12 horas y para el año 2013(obtenidos entre el 12vo y 16 vo mes despues de la implantacion del sistema COREBANK se obtuvo como promedio 25,52 horas, lo que representa una disminucion de 20,6 horas en la evaluacion de los creditos).
Para validar la hipotesis especifica: "La implementación de un sistema de información basado en un enfoque de procesos, disminuye el tiempo de evaluacion y otorgamiento de creditos de la Mircrofinanciera CRECER", se realizo una prueba estadistica usando la distribucion t-student para realizar la comparación se tomo datos del primer semestre del año 2012 (con media poblacional igual a 46.12 horas) y los datos correspondientes al segundo semestre del 2013.

\begin{figure}[H]
    \includegraphics[width=\textwidth]{Imagenes_Parte_4.2/image2.jpeg}
    \caption{Zona de Rechazo y no rechazo de hipotesis (Tiempo de evaluacion y otorgamiento de creditos)}
\end{figure}

En la grafica se aprecia la zona de aceptacion y las zonas de rechazo de la hipotesis nula, dividas por los puntos criticos -2.7765 y 2.7765, tambien se observa el valor del estadístico de prueba(-413.27), ubicado en la zona de rechazo de la Ho del lado izquierdo.

Por ello se toma la decision de \textbf{rechazar la hipotesis nula y aceptar la hipotesis alternativa}, en conclusion se puede decir que existe evidencia suficiente para afirmar que el tiempo de evaluacion y otorgamiento de creditos disminuyo el segundo semestre del año 2013 con respecto al primer semestre del año 2012.

\subsubsection{Validacion de resultados del indicador grado de satisfaccion del cliente}
Para validar los resultados del indicador de sdatisfccion del cliente es necesario procesar las encuestas tomadas en los años 2012 y 2013 y analizar si las puntuacion de estas se incrementaron, disminuyeron o se mantuvieron iguales, cada encuestra estaba compuesta por 5 preguntas.
\begin{figure}[H]
    \includegraphics[width=\textwidth]{Imagenes_Parte_4.2/image3.jpeg}
    \caption{Comparacion de las encuestas a los clientes 2012-2013)}
\end{figure}
En la grafica se aprecia un cuadro comparativo entre los promedios de los grados valoracion obtenidos en las encuestas tomadas el 2012, orientadas al SI HORMIGA; y las tomadas el 2013, orientados al SI CoreBank.
Para validar la hipotesis especifica: "La implementacion de un sistema de informacion basado en un enfoque de procesos, incrementa el grado de satisfaccion de los clientes con respecto al servicio de creditos que brinda la Microfinanciera CRECER", se realizo la prueba estadistica utilizado la distribucion normal, para realizar la comparacion se tomaron 98 encuestas realizadas en junio del año 2012 y las 98 encuestas tomadas en octubre del año 2013.


\begin{figure}[h]
    \centering
    \includegraphics[width=\textwidth]{Imagenes_Parte_4.2/image4.jpeg}
    \label{Maps}
    \caption{Zona de rechazo y no rechazo de hipotesis (Grado de satisfaccion del cliente) }
   \end{figure}

Por ello se toma la decision de \textbf{rechazar la hipotesis nula y aceptar la hipotesis alternativa}, en conclusion se puede decir que existe evidencia suficiente para afirmar que el grado de satisfaccion de los clientes con respecto al servicio de creditos que brinda la Microfinanciera CRECER se incremento en el año 2013 con respecto al año 2012.

\subsubsection{Validacion de resultados del indicador grado de satisfaccion del personal}
A continuacion se presentan 2 graficos comparativos donde se resume los resultados de las encuestas tomados en el 2012 y 2013, y se muestran los promedios obtenidos, despues de dar un puntaje a cada una de las respuestas
\begin{figure}[H]
    \includegraphics[width=\textwidth]{Imagenes_Parte_4.2/image5.jpeg}
    \caption{Comparacion de las encuestas al personal 2012-2013}
\end{figure}
En la grafica se aprecia que el personal indica que el sistema mas facil e intuitivo de usar es el COREBANK, dandole una puntiacion mas alta frente al sistema a HORMIGAen todas las preguntas realizadas.
Para validar la hipotesis especifica: "La implementacion de un sistema de informacion basado en un enfoque de procesos, incrementa el grado de satisfaccion del personal con respecto al apoyo que recibe por parte del sistema de informacion, se realizo la prueba estadistica utilizado la distribucion normal, para realizar la comparacion se tomaron 43 encuestas realizadas en junio del año 2012 y las 50 encuestas tomadas en octubre del año 2013.
\begin{figure}[H]
    \includegraphics[width=\textwidth]{Imagenes_Parte_4.2/image6.jpeg}
    \caption{Zona de rechazo y no rechazo de hipotesis(Grado de satisfaccion del personal)}
\end{figure}
Por todo lo evidenciado anteriormente se concluye que las Hipotesis especificas y las hipotesis general son Verdaderas, ya que la implementacion del sistema de informacion basado en un enfoque de procesos disminuyo el porcentaje de morosidad, redujo los tiempos empleados en la evaluacion de los creditos, incremento el grado de satisfaccion de los clientes con respecto al servicio que reciben por parte de la Microfinanciera e incremento el grado de satisfaccion del personal con respecto al apoyo que reciben del sistema de informacion , por lo tanto mejoro la operatividad del area de creditos de la Microfinanciera CRECER.
\end{document}
